\documentclass{book}
\usepackage[french]{babel}
\usepackage{mathtools}
\title{''Monier MPSI 3eme edition''}
\author{''Sophie''}

 
\begin{document} 

 \chapter{Vocabulaire de la théorie des ensembles}
\section{Ensembles}
 \subsection{Exercice}
\subsubsection{a}
 \[p\Longleftrightarrow q\]
\[\Longleftrightarrow\]


\[
\left \{
\begin{array}{c}
     p \Longrightarrow q \\
    q \Longrightarrow p
\end{array}
\right.
\]
Comme \(P \land Q \) a la même table de vérité que \(Q \land P\), on a 
 \[
\left \{
\begin{array}{c}
     p \Longrightarrow q \\
    q \Longrightarrow p
\end{array}
\right.
\]
\[\Longleftrightarrow\]
 \[
\left \{
\begin{array}{c}
     q \Longrightarrow p \\
    p \Longrightarrow q
\end{array}
\right.
\]
d'où
\[q \Longleftrightarrow p\]
\subsubsection{b}
On a juste besoin de démontrer que \(q\Longrightarrow p\). Le reste se déduit par permutation circulaire.

Comme \(q\Longrightarrow r\) et \(r \Longrightarrow p\) alors par transitivité de l'implication, on a \(q\Longrightarrow p\).
\subsubsection{c}


 
\end{document}